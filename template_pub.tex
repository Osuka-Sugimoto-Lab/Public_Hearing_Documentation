\RequirePackage{plautopatch}
\RequirePackage[l2tabu, orthodox]{nag}

\documentclass[uplatex,dvipdfmx]{jlreq}
\usepackage{osukalab_pub_doc}
\usepackage{epic,eepic}
\usepackage{graphicx}
\usepackage{amsmath}
\usepackage[all, warning]{onlyamsmath}
\usepackage{siunitx}
\usepackage{txfonts}
\usepackage{amssymb}
\usepackage{cite}

\pagestyle{empty}

\headding{修士論文要旨}%修論の場合はこっち
%\headding{卒業論文要旨}%卒論の場合はこっち

\date{2025年2月7日} %日付確認
\title{論文公聴会資料(タイトルと差し替える)} 

\author{機械工学専攻 大須賀・杉本・石原研究室 氏名}

\begin{document}

\maketitle
\thispagestyle{empty}
\section{緒言}
ここに本文を書く.

\begin{figure}[t]
    \centering
    \includegraphics[width=\columnwidth]{./figure/testfig.pdf}
    \caption{Test figure. 象の鼻は長いぞう象の鼻は長いぞう象の鼻は長いぞう象の鼻は長いぞう}
    \label{fig:test}
\end{figure}


\subsection{節}

\begin{table}[t]
    \centering
    \caption{Test table. 象の鼻は長いぞう象の鼻は長いぞう象の鼻は長いぞう象の鼻は長いぞう}
    \begin{tabular}{ccc}
        \hline
        A & B & C \\
        \hline
        \hline
        1 & 2 & 3 \\
        4 & 5 & 6 \\
        \hline
    \end{tabular}
    \label{tab:test}
\end{table}

参考文献のテスト\cite{bibtest,mike}.


{\small
%\bibliographystyle{osukalab}
\bibliographystyle{junsrt}
\bibliography{myref}
\clearpage
\end{document}
}
